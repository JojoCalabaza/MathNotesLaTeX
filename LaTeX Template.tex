\documentclass{article}

\usepackage{tcolorbox, varwidth} % Boxes
\usepackage{amsthm} % For the proof environment

\tcbuselibrary{breakable}
\tcbuselibrary{theorems}
\tcbuselibrary{skins}

% Colours
\definecolor{TheoremColor}{RGB}{34,139,34} % Green
\definecolor{DefColor}{RGB}{45, 52, 151} % Blue
\definecolor{CorollaryColor}{RGB}{158, 80, 143} % Purple
\definecolor{ExampleColor}{RGB}{226,135,67} % Orange
\definecolor{ProofColor}{RGB}{0,177,160} % Aqua


\title{Maths Notes With \LaTeX}
\author{PolyMaths}

% Theorem
\newtcbtheorem[number within = section]{theorem}{Theorem}%
{enhanced,frame empty,interior empty,colframe=TheoremColor!50!white,
	coltitle=TheoremColor!50!black,fonttitle=\bfseries,colbacktitle=TheoremColor!15!white,
	borderline={0.5mm}{0mm}{TheoremColor!15!white},
	borderline={0.5mm}{0mm}{TheoremColor!50!white,dashed},
	attach boxed title to top center={yshift=-2mm},
	boxed title style={boxrule=0.4pt},varwidth boxed title}{theo}

% Definition
\newtcbtheorem[number within = section]{definition}{Definition}%
{enhanced,frame empty,interior empty,colframe=DefColor!50!white,
	coltitle=DefColor!50!black,fonttitle=\bfseries,colbacktitle=DefColor!15!white,
	borderline={0.5mm}{0mm}{DefColor!15!white},
	borderline={0.5mm}{0mm}{DefColor!50!white,dashed},
	attach boxed title to top center={yshift=-2mm},
	boxed title style={boxrule=0.4pt},varwidth boxed title}{defo}

% Corollary
\newtcbtheorem[number within = section]{corollary}{Corollary}%
{enhanced,frame empty,interior empty,colframe=CorollaryColor!50!white,
	coltitle=CorollaryColor!50!black,fonttitle=\bfseries,colbacktitle=CorollaryColor!15!white,
	borderline={0.5mm}{0mm}{CorollaryColor!15!white},
	borderline={0.5mm}{0mm}{CorollaryColor!50!white,dashed},
	attach boxed title to top center={yshift=-2mm},
	boxed title style={boxrule=0.4pt},varwidth boxed title}{defo}

% Example
\newtcbtheorem[number within = section]{example}{Example}%
{enhanced,frame empty,interior empty,colframe=ExampleColor!50!white,
	coltitle=ExampleColor!50!black,fonttitle=\bfseries,colbacktitle=ExampleColor!15!white,
	borderline={0.5mm}{0mm}{ExampleColor!15!white},
	borderline={0.5mm}{0mm}{ExampleColor!50!white,dashed},
	attach boxed title to top center={yshift=-2mm},
	boxed title style={boxrule=0.4pt},varwidth boxed title}{defo}

% Proof
\tcolorboxenvironment{proof}{% `proof' from `amsthm'
	blanker,breakable,left=5mm,
	before skip=10pt,after skip=10pt,
	borderline west={1mm}{0pt}{ProofColor!50!white}}

\begin{document}
	\maketitle
	\section{Introduction}
	Welcome to the template! Let's add a definition.
	\begin{definition}{Limit of a function}{}
	If, for every $ \epsilon >0 $ there exists some $ \delta >0 $ such $ 0<|x-a|<\delta $ implies $ |f(x)-L|<\epsilon $ then we say that the function $ f $ has a limit of $ L $ at $ a $ and we write
	\[\lim_{x\to a}f(x)=L.\]
	\end{definition}
	And let's also follow it up with a theorem:
	\begin{theorem}{Fermat's Last Theorem}{}
		The equation
		\[a^n+b^n=c^n\]
		has no integer solutions for every integer $ n> 2 $.
	\end{theorem}
	\begin{proof}
		I have discovered a truly marvellous proof of this, which this margin is too narrow to contain.
	\end{proof}
	But this immediately implies the following corollary:
	\begin{corollary}{Riemann's}{}
		Every non-trivial zero of the Riemann $ \zeta $ function has real part one-half.
	\end{corollary}
	Which we can demonstrate with an example:
	\begin{example}{Poincare}{}
		Consider a simply connected, closed 3-manifold. Notice that it is homeomorphic to the 3-sphere!
	\end{example}
		
\end{document}















